% Options for packages loaded elsewhere
% Options for packages loaded elsewhere
\PassOptionsToPackage{unicode}{hyperref}
\PassOptionsToPackage{hyphens}{url}
%
\documentclass[
  letterpaper,
]{book}
\usepackage{xcolor}
\usepackage{amsmath,amssymb}
\setcounter{secnumdepth}{5}
\usepackage{iftex}
\ifPDFTeX
  \usepackage[T1]{fontenc}
  \usepackage[utf8]{inputenc}
  \usepackage{textcomp} % provide euro and other symbols
\else % if luatex or xetex
  \usepackage{unicode-math} % this also loads fontspec
  \defaultfontfeatures{Scale=MatchLowercase}
  \defaultfontfeatures[\rmfamily]{Ligatures=TeX,Scale=1}
\fi
\usepackage{lmodern}
\ifPDFTeX\else
  % xetex/luatex font selection
\fi
% Use upquote if available, for straight quotes in verbatim environments
\IfFileExists{upquote.sty}{\usepackage{upquote}}{}
\IfFileExists{microtype.sty}{% use microtype if available
  \usepackage[]{microtype}
  \UseMicrotypeSet[protrusion]{basicmath} % disable protrusion for tt fonts
}{}
\makeatletter
\@ifundefined{KOMAClassName}{% if non-KOMA class
  \IfFileExists{parskip.sty}{%
    \usepackage{parskip}
  }{% else
    \setlength{\parindent}{0pt}
    \setlength{\parskip}{6pt plus 2pt minus 1pt}}
}{% if KOMA class
  \KOMAoptions{parskip=half}}
\makeatother
% Make \paragraph and \subparagraph free-standing
\makeatletter
\ifx\paragraph\undefined\else
  \let\oldparagraph\paragraph
  \renewcommand{\paragraph}{
    \@ifstar
      \xxxParagraphStar
      \xxxParagraphNoStar
  }
  \newcommand{\xxxParagraphStar}[1]{\oldparagraph*{#1}\mbox{}}
  \newcommand{\xxxParagraphNoStar}[1]{\oldparagraph{#1}\mbox{}}
\fi
\ifx\subparagraph\undefined\else
  \let\oldsubparagraph\subparagraph
  \renewcommand{\subparagraph}{
    \@ifstar
      \xxxSubParagraphStar
      \xxxSubParagraphNoStar
  }
  \newcommand{\xxxSubParagraphStar}[1]{\oldsubparagraph*{#1}\mbox{}}
  \newcommand{\xxxSubParagraphNoStar}[1]{\oldsubparagraph{#1}\mbox{}}
\fi
\makeatother


\usepackage{longtable,booktabs,array}
\usepackage{calc} % for calculating minipage widths
% Correct order of tables after \paragraph or \subparagraph
\usepackage{etoolbox}
\makeatletter
\patchcmd\longtable{\par}{\if@noskipsec\mbox{}\fi\par}{}{}
\makeatother
% Allow footnotes in longtable head/foot
\IfFileExists{footnotehyper.sty}{\usepackage{footnotehyper}}{\usepackage{footnote}}
\makesavenoteenv{longtable}
\usepackage{graphicx}
\makeatletter
\newsavebox\pandoc@box
\newcommand*\pandocbounded[1]{% scales image to fit in text height/width
  \sbox\pandoc@box{#1}%
  \Gscale@div\@tempa{\textheight}{\dimexpr\ht\pandoc@box+\dp\pandoc@box\relax}%
  \Gscale@div\@tempb{\linewidth}{\wd\pandoc@box}%
  \ifdim\@tempb\p@<\@tempa\p@\let\@tempa\@tempb\fi% select the smaller of both
  \ifdim\@tempa\p@<\p@\scalebox{\@tempa}{\usebox\pandoc@box}%
  \else\usebox{\pandoc@box}%
  \fi%
}
% Set default figure placement to htbp
\def\fps@figure{htbp}
\makeatother


% definitions for citeproc citations
\NewDocumentCommand\citeproctext{}{}
\NewDocumentCommand\citeproc{mm}{%
  \begingroup\def\citeproctext{#2}\cite{#1}\endgroup}
\makeatletter
 % allow citations to break across lines
 \let\@cite@ofmt\@firstofone
 % avoid brackets around text for \cite:
 \def\@biblabel#1{}
 \def\@cite#1#2{{#1\if@tempswa , #2\fi}}
\makeatother
\newlength{\cslhangindent}
\setlength{\cslhangindent}{1.5em}
\newlength{\csllabelwidth}
\setlength{\csllabelwidth}{3em}
\newenvironment{CSLReferences}[2] % #1 hanging-indent, #2 entry-spacing
 {\begin{list}{}{%
  \setlength{\itemindent}{0pt}
  \setlength{\leftmargin}{0pt}
  \setlength{\parsep}{0pt}
  % turn on hanging indent if param 1 is 1
  \ifodd #1
   \setlength{\leftmargin}{\cslhangindent}
   \setlength{\itemindent}{-1\cslhangindent}
  \fi
  % set entry spacing
  \setlength{\itemsep}{#2\baselineskip}}}
 {\end{list}}
\usepackage{calc}
\newcommand{\CSLBlock}[1]{\hfill\break\parbox[t]{\linewidth}{\strut\ignorespaces#1\strut}}
\newcommand{\CSLLeftMargin}[1]{\parbox[t]{\csllabelwidth}{\strut#1\strut}}
\newcommand{\CSLRightInline}[1]{\parbox[t]{\linewidth - \csllabelwidth}{\strut#1\strut}}
\newcommand{\CSLIndent}[1]{\hspace{\cslhangindent}#1}



\setlength{\emergencystretch}{3em} % prevent overfull lines

\providecommand{\tightlist}{%
  \setlength{\itemsep}{0pt}\setlength{\parskip}{0pt}}



 


\usepackage{emoji}
\makeatletter
\@ifpackageloaded{tcolorbox}{}{\usepackage[skins,breakable]{tcolorbox}}
\@ifpackageloaded{fontawesome5}{}{\usepackage{fontawesome5}}
\definecolor{quarto-callout-color}{HTML}{909090}
\definecolor{quarto-callout-note-color}{HTML}{0758E5}
\definecolor{quarto-callout-important-color}{HTML}{CC1914}
\definecolor{quarto-callout-warning-color}{HTML}{EB9113}
\definecolor{quarto-callout-tip-color}{HTML}{00A047}
\definecolor{quarto-callout-caution-color}{HTML}{FC5300}
\definecolor{quarto-callout-color-frame}{HTML}{acacac}
\definecolor{quarto-callout-note-color-frame}{HTML}{4582ec}
\definecolor{quarto-callout-important-color-frame}{HTML}{d9534f}
\definecolor{quarto-callout-warning-color-frame}{HTML}{f0ad4e}
\definecolor{quarto-callout-tip-color-frame}{HTML}{02b875}
\definecolor{quarto-callout-caution-color-frame}{HTML}{fd7e14}
\makeatother
\makeatletter
\@ifpackageloaded{bookmark}{}{\usepackage{bookmark}}
\makeatother
\makeatletter
\@ifpackageloaded{caption}{}{\usepackage{caption}}
\AtBeginDocument{%
\ifdefined\contentsname
  \renewcommand*\contentsname{Table of contents}
\else
  \newcommand\contentsname{Table of contents}
\fi
\ifdefined\listfigurename
  \renewcommand*\listfigurename{List of Figures}
\else
  \newcommand\listfigurename{List of Figures}
\fi
\ifdefined\listtablename
  \renewcommand*\listtablename{List of Tables}
\else
  \newcommand\listtablename{List of Tables}
\fi
\ifdefined\figurename
  \renewcommand*\figurename{Figure}
\else
  \newcommand\figurename{Figure}
\fi
\ifdefined\tablename
  \renewcommand*\tablename{Table}
\else
  \newcommand\tablename{Table}
\fi
}
\@ifpackageloaded{float}{}{\usepackage{float}}
\floatstyle{ruled}
\@ifundefined{c@chapter}{\newfloat{codelisting}{h}{lop}}{\newfloat{codelisting}{h}{lop}[chapter]}
\floatname{codelisting}{Listing}
\newcommand*\listoflistings{\listof{codelisting}{List of Listings}}
\makeatother
\makeatletter
\makeatother
\makeatletter
\@ifpackageloaded{caption}{}{\usepackage{caption}}
\@ifpackageloaded{subcaption}{}{\usepackage{subcaption}}
\makeatother
\makeatletter
\@ifpackageloaded{sidenotes}{}{\usepackage{sidenotes}}
\@ifpackageloaded{marginnote}{}{\usepackage{marginnote}}
\makeatother
\usepackage{bookmark}
\IfFileExists{xurl.sty}{\usepackage{xurl}}{} % add URL line breaks if available
\urlstyle{same}
\hypersetup{
  pdftitle={Líneas de Producto Software},
  pdfauthor={José Miguel Horcas},
  hidelinks,
  pdfcreator={LaTeX via pandoc}}


\title{Líneas de Producto Software}
\author{José Miguel Horcas}
\date{2026-01-16}
\begin{document}
\frontmatter
\maketitle

\renewcommand*\contentsname{Table of contents}
{
\setcounter{tocdepth}{2}
\tableofcontents
}

\mainmatter
\bookmarksetup{startatroot}

\chapter*{Preface}\label{preface}
\addcontentsline{toc}{chapter}{Preface}

\markboth{Preface}{Preface}

This is a Quarto book.

To learn more about Quarto books visit
\url{https://quarto.org/docs/books}.

\part{Líneas de Productos}

\section*{Introducción a las Líneas de
Productos}\label{introducciuxf3n-a-las-luxedneas-de-productos}
\addcontentsline{toc}{section}{Introducción a las Líneas de Productos}

\markright{Introducción a las Líneas de Productos}

\begin{quote}
``Un cliente puede elegir cualquier color para su coche, siempre que sea
negro.''

--- \emph{Henry Ford, 1922}
\end{quote}

La célebre afirmación de Ford y Crowther
\citeproc{ref-Ford1922_MyLifeAndWork--707ddefda8b9d501a1d63dad3fcb74d0cb28b51a}{{[}1{]}}
resume de forma provocadora la filosofía de la producción en masa de
principios del siglo XX. Lejos de ser una simple exageración, esta frase
tenía un fundamento estrictamente técnico: el color negro era el que se
secaba con mayor rapidez.

Esta ventaja logística permitía acelerar la cadena de montaje y reducir
los costes al mínimo, a costa de eliminar cualquier posibilidad de
elección para el cliente. Se trataba de una producción eficiente, pero
rígida, basada en la estandarización absoluta.

Este escenario representa el extremo opuesto al concepto de
\textbf{Línea de Productos}. Mientras que Ford sacrificó la variabilidad
en favor de la eficiencia, las líneas de productos persiguen
precisamente lo contrario: \textbf{gestionar la variabilidad de forma
sistemática y controlada, sin renunciar a la eficiencia.}

En una Línea de Productos moderna, el objetivo es alcanzar la llamada
\textbf{personalización masiva}: combinar la eficiencia de la producción
en serie con la capacidad de ofrecer productos adaptados a las
necesidades de cada cliente. En otras palabras, aspiramos a mantener la
eficiencia del modelo de Ford, pero permitiendo que cada cliente pueda,
ahora sí, elegir el color de su coche.

\section*{Referencias}\label{referencias}

\markright{Referencias}

\phantomsection\label{refs--707ddefda8b9d501a1d63dad3fcb74d0cb28b51a}
\begin{CSLReferences}{0}{0}
\bibitem[\citeproctext]{ref-Ford1922_MyLifeAndWork--707ddefda8b9d501a1d63dad3fcb74d0cb28b51a}
\CSLLeftMargin{{[}1{]} }%
\CSLRightInline{H. Ford and S. Crowther, \emph{My life and work}. Garden
City, NY: Doubleday, Page \& Company, 1922.}

\end{CSLReferences}

\chapter{Paradigmas de producción}\label{paradigmas-de-producciuxf3n}

\emph{De la artesanía a la personalización masiva: Evolución histórica
de la producción y la variabilidad}

Para comprender qué es una \textbf{Línea de Productos} (\emph{Product
Line}, PL) y por qué se ha convertido en un pilar de la ingeniería y del
desarrollo de software moderno, conviene dar un paso atrás y observar
cómo ha evolucionado históricamente la forma en que producimos bienes.

Desde una simple silla de madera hasta los sistemas software más
complejos, la producción ha perseguido siempre un equilibrio ---nunca
trivial--- entre \textbf{coste}, \textbf{eficiencia} (velocidad) y
\textbf{personalización}. A lo largo del tiempo, este equilibrio ha dado
lugar a tres grandes \textbf{paradigmas de producción}, que sirven como
marco conceptual para entender las líneas de productos:

\begin{itemize}
\tightlist
\item
  \textbf{Artesanía:} máxima personalización a costa de la eficiencia y
  la escalabilidad.
\item
  \textbf{Producción en masa:} eficiencia mediante estandarización y
  reducción de costes, sacrificando la personalización.
\item
  \textbf{Personalización masiva:} combinación de eficiencia y
  flexibilidad, gestionando la variabilidad de manera controlada.
\end{itemize}

\begin{figure}

\centering{

\includegraphics[width=1\linewidth,height=\textheight,keepaspectratio]{index_files/mediabag/01-part1/ch01-intro/images/produccion.pdf}

}

\caption{\label{fig-produccion}Paradigmas de producción y su relación
con coste, eficiencia y personalización.}

\end{figure}%

\begin{center}\rule{0.5\linewidth}{0.5pt}\end{center}

\section{Artesanía}\label{artesanuxeda}

Antes de la Revolución Industrial (siglos XVIII y XIX), la producción
era fundamentalmente \textbf{artesanal}. En este contexto, cada producto
se diseñaba y fabricaba de manera individual, atendiendo a las
necesidades específicas de un cliente concreto.

\begin{tcolorbox}[enhanced jigsaw, bottomtitle=1mm, leftrule=.75mm, arc=.35mm, coltitle=black, colframe=quarto-callout-note-color-frame, title=\textcolor{quarto-callout-note-color}{\faInfo}\hspace{0.5em}{Definición}, toprule=.15mm, breakable, opacitybacktitle=0.6, opacityback=0, colback=white, colbacktitle=quarto-callout-note-color!10!white, left=2mm, rightrule=.15mm, toptitle=1mm, titlerule=0mm, bottomrule=.15mm]

\textbf{Producción artesanal (\emph{Handcrafting})}\\
Modelo de producción basado en la fabricación individual de cada
producto, altamente personalizado, pero con baja eficiencia y escasa
capacidad de escalado.

\end{tcolorbox}

Este modelo se distingue por ofrecer un grado de personalización total y
una alta calidad, estrechamente ligada a la destreza y experiencia del
artesano. A cambio, presenta importantes limitaciones: el proceso es
lento, costoso y difícilmente escalable, ya que cada unidad debe
producirse de forma independiente.

\begin{tcolorbox}[enhanced jigsaw, bottomtitle=1mm, leftrule=.75mm, arc=.35mm, coltitle=black, colframe=quarto-callout-tip-color-frame, title=\textcolor{quarto-callout-tip-color}{\faLightbulb}\hspace{0.5em}{Ejemplo}, toprule=.15mm, breakable, opacitybacktitle=0.6, opacityback=0, colback=white, colbacktitle=quarto-callout-tip-color!10!white, left=2mm, rightrule=.15mm, toptitle=1mm, titlerule=0mm, bottomrule=.15mm]

Si alguien deseaba una mesa, el artesano la concebía y construía
exclusivamente para ese cliente, eligiendo los materiales, las
dimensiones y los acabados de forma completamente personalizada. No
existían dos mesas idénticas, ni procesos fácilmente reutilizables.

\end{tcolorbox}

\begin{marginfigure}

\centering{

\includegraphics[width=1\linewidth,height=\textheight,keepaspectratio]{01-part1/ch01-intro/images/artesania.png}

}

\caption{\label{fig-artesania}Artesanía.}

\end{marginfigure}%

En el ámbito del software, este enfoque equivale a desarrollar un
sistema \textbf{desde cero} para un único cliente, sin reutilizar
componentes, arquitecturas ni soluciones previas. Aunque viable en
contextos muy concretos, resulta difícilmente sostenible a gran escala
cuando se requiere producir múltiples sistemas de forma eficiente.

\begin{center}\rule{0.5\linewidth}{0.5pt}\end{center}

\section{Producción en masa}\label{producciuxf3n-en-masa}

Con la llegada de la Revolución Industrial y el trabajo pionero de Henry
Ford a comienzos del siglo XX
\citeproc{ref-Ford1922_MyLifeAndWork--2}{{[}1{]}}, surge la necesidad de
fabricar \textbf{muchos productos, rápidamente y a bajo coste}. La
respuesta a este desafío fue la \textbf{producción en masa}
\citeproc{ref-Tseng2001_MassCustomization--2}{{[}2{]}}.

\begin{tcolorbox}[enhanced jigsaw, bottomtitle=1mm, leftrule=.75mm, arc=.35mm, coltitle=black, colframe=quarto-callout-note-color-frame, title=\textcolor{quarto-callout-note-color}{\faInfo}\hspace{0.5em}{Definición}, toprule=.15mm, breakable, opacitybacktitle=0.6, opacityback=0, colback=white, colbacktitle=quarto-callout-note-color!10!white, left=2mm, rightrule=.15mm, toptitle=1mm, titlerule=0mm, bottomrule=.15mm]

\textbf{Producción en masa \emph{(Mass Production)}}\\
Modelo de producción orientado a la eficiencia y al bajo coste mediante
la estandarización de procesos y componentes, sacrificando la
personalización del producto final.

\end{tcolorbox}

Este paradigma se sustenta en dos ideas fundamentales:

\begin{enumerate}
\def\labelenumi{\arabic{enumi}.}
\tightlist
\item
  \textbf{Líneas de ensamblaje:} el proceso productivo se divide en
  tareas simples, repetitivas y altamente especializadas.
\item
  \textbf{Componentes estandarizados:} se diseñan piezas idénticas,
  intercambiables y reutilizables en todos los productos.
\end{enumerate}

Estas ideas permitieron reducir drásticamente los costes de producción,
acortar los tiempos de fabricación y mejorar la calidad media gracias al
control sistemático de cada componente. El éxito de este modelo tuvo, no
obstante, una consecuencia clara: la \textbf{personalización se vio
severamente limitada}. Todos los productos eran esencialmente iguales, y
el cliente apenas tenía capacidad de elección.

\begin{tcolorbox}[enhanced jigsaw, bottomtitle=1mm, leftrule=.75mm, arc=.35mm, coltitle=black, colframe=quarto-callout-tip-color-frame, title=\textcolor{quarto-callout-tip-color}{\faLightbulb}\hspace{0.5em}{Ejemplo}, toprule=.15mm, breakable, opacitybacktitle=0.6, opacityback=0, colback=white, colbacktitle=quarto-callout-tip-color!10!white, left=2mm, rightrule=.15mm, toptitle=1mm, titlerule=0mm, bottomrule=.15mm]

En una fábrica de automóviles, cada vehículo se ensambla siguiendo
exactamente el mismo proceso y utilizando las mismas piezas. El
resultado es un producto fiable y económico, pero prácticamente
indistinguible del resto.

\end{tcolorbox}

\begin{marginfigure}

\centering{

\includegraphics[width=1\linewidth,height=\textheight,keepaspectratio]{01-part1/ch01-intro/images/produccion_masa.png}

}

\caption{\label{fig-produccion_masa}Producción en masa.}

\end{marginfigure}%

Un paralelismo claro en el mundo del software es el \textbf{software
estándar} o \emph{off-the-shelf}: soluciones de ``talla única''
(\emph{one-size-fits-all}) como Microsoft Word, SAP o Windows. Son
productos robustos y ampliamente probados, pero todos los usuarios
reciben exactamente el mismo sistema, sin posibilidad de adaptación
individual.

\begin{center}\rule{0.5\linewidth}{0.5pt}\end{center}

\section{Personalización masiva}\label{personalizaciuxf3n-masiva}

A finales del siglo XX, y especialmente con la irrupción de Internet, el
mercado deja de conformarse con productos uniformes. Los clientes desean
conservar las ventajas de la producción en masa ---rapidez y bajo
coste---, pero exigen al mismo tiempo \textbf{productos adaptados a sus
necesidades específicas}.

\begin{tcolorbox}[enhanced jigsaw, bottomtitle=1mm, leftrule=.75mm, arc=.35mm, coltitle=black, colframe=quarto-callout-note-color-frame, title=\textcolor{quarto-callout-note-color}{\faInfo}\hspace{0.5em}{Definición}, toprule=.15mm, breakable, opacitybacktitle=0.6, opacityback=0, colback=white, colbacktitle=quarto-callout-note-color!10!white, left=2mm, rightrule=.15mm, toptitle=1mm, titlerule=0mm, bottomrule=.15mm]

\textbf{Personalización masiva \emph{(Mass Customization)}}\\
Modelo de producción que combina la eficiencia de la estandarización con
la capacidad de ofrecer productos personalizados mediante mecanismos de
configuración controlada.

\end{tcolorbox}

La idea central de este paradigma puede resumirse como la combinación de
\textbf{estandarización y flexibilidad}. En la práctica, el núcleo del
producto (su estructura y componentes base) se produce de forma uniforme
y eficiente, mientras que la variabilidad se introduce de manera
controlada en la fase de configuración final.

Este enfoque se apoya en el uso de \textbf{configuradores} o sistemas de
opciones, que permiten seleccionar características dentro de un catálogo
predefinido. De este modo, el cliente obtiene un producto personalizado
sin que el productor renuncie a los beneficios de la producción en masa.

\begin{tcolorbox}[enhanced jigsaw, bottomtitle=1mm, leftrule=.75mm, arc=.35mm, coltitle=black, colframe=quarto-callout-tip-color-frame, title=\textcolor{quarto-callout-tip-color}{\faLightbulb}\hspace{0.5em}{Ejemplo}, toprule=.15mm, breakable, opacitybacktitle=0.6, opacityback=0, colback=white, colbacktitle=quarto-callout-tip-color!10!white, left=2mm, rightrule=.15mm, toptitle=1mm, titlerule=0mm, bottomrule=.15mm]

Al adquirir un automóvil, el chasis y el motor se fabrican en masa
siguiendo procesos altamente estandarizados. Sin embargo, el comprador
puede elegir el color, la tapicería o el sistema de sonido, configurando
un vehículo adaptado a sus preferencias personales.

\end{tcolorbox}

\begin{marginfigure}

\centering{

\includegraphics[width=1\linewidth,height=\textheight,keepaspectratio]{01-part1/ch01-intro/images/personalizacion_masiva.png}

}

\caption{\label{fig-personalizacion_masiva}Personalización masiva.}

\end{marginfigure}%

En el ámbito del software, este mismo principio permite ofrecer sistemas
altamente configurables a partir de una base común. El concepto de
\textbf{Línea de Productos Software} constituye precisamente la
aplicación sistemática de la personalización masiva al desarrollo de
software.

\section*{Referencias}\label{referencias-1}

\markright{Referencias}

\phantomsection\label{refs--2}
\begin{CSLReferences}{0}{0}
\bibitem[\citeproctext]{ref-Ford1922_MyLifeAndWork--2}
\CSLLeftMargin{{[}1{]} }%
\CSLRightInline{H. Ford and S. Crowther, \emph{My life and work}. Garden
City, NY: Doubleday, Page \& Company, 1922.}

\bibitem[\citeproctext]{ref-Tseng2001_MassCustomization--2}
\CSLLeftMargin{{[}2{]} }%
\CSLRightInline{M. M. Tseng and J. Jiao, {``Mass {Customization},''} in
\emph{Handbook of {Industrial} {Engineering}}, John Wiley \& Sons, Ltd,
2001, pp. 684--709. doi:
\href{https://doi.org/10.1002/9780470172339.ch25}{10.1002/9780470172339.ch25}.}

\end{CSLReferences}

\chapter{Línea de Productos}\label{luxednea-de-productos}

**

Tras el recorrido por la evolución de los paradigmas de producción
---desde la artesanía hasta la personalización masiva---, podemos ya
poner nombre a la estrategia que permite alcanzar un equilibrio
sostenible entre eficiencia y personalización.

Esa estrategia recibe el nombre de \textbf{Línea de Productos} y
constituye uno de los conceptos centrales de la ingeniería moderna, y en
particular del desarrollo de software basado en la reutilización
sistemática.

\begin{tcolorbox}[enhanced jigsaw, bottomtitle=1mm, leftrule=.75mm, arc=.35mm, coltitle=black, colframe=quarto-callout-note-color-frame, title=\textcolor{quarto-callout-note-color}{\faInfo}\hspace{0.5em}{Definición}, toprule=.15mm, breakable, opacitybacktitle=0.6, opacityback=0, colback=white, colbacktitle=quarto-callout-note-color!10!white, left=2mm, rightrule=.15mm, toptitle=1mm, titlerule=0mm, bottomrule=.15mm]

\textbf{Línea de Productos \emph{(Product Line, PL)}}\\
Conjunto de productos de un mismo fabricante que comparten similitudes
sustanciales y que se construyen, idealmente, a partir de un conjunto
común de partes reutilizables.

\end{tcolorbox}

En lugar de diseñar un único producto estandarizado ---la ``talla
única'' propia de la producción en masa---, una Línea de Productos
persigue atender distintos segmentos del mercado a partir de una base
común. La clave está en una estrategia explícita de reutilización:

\begin{enumerate}
\def\labelenumi{\arabic{enumi}.}
\tightlist
\item
  Se desarrolla un núcleo común de activos reutilizables
  (\textbf{\emph{core assets}}).
\item
  Se identifican los puntos donde los productos deben diferir.
\item
  Se definen variantes controladas que permiten adaptar el producto
  final.
\end{enumerate}

De este modo, no se construyen productos independientes, sino
configuraciones distintas de una misma familia.

\begin{quote}
\textbf{Línea de Productos = Reutilización sistemática + Capacidad de
variación}
\end{quote}

Esta es, en esencia, la materialización de la personalización masiva
aplicada a la ingeniería.

\section{Ejemplos}\label{ejemplos}

\subsection{El configurador de automóviles
🚗}\label{el-configurador-de-automuxf3viles}

La industria automotriz es uno de los ejemplos más representativos de
Líneas de Productos.

Un fabricante como BMW no ofrece un único coche, sino un portafolio de
modelos: berlinas, familiares, deportivos o eléctricos. Aunque difieren
en precio, prestaciones o apariencia, comparten una parte sustancial de
su ingeniería.

Partes reutilizables: plataformas de chasis, componentes electrónicos,
sistemas de infoentretenimiento, software embarcado.

Variabilidad controlada: el cliente utiliza un configurador para
seleccionar motor, color, equipamiento o interiores.

El sistema ensambla un vehículo concreto a partir de una base común y un
conjunto de opciones válidas.

\marginnote{\begin{footnotesize}

El configurador actúa como una interfaz de la variabilidad.

\end{footnotesize}}

El resultado es doble: el fabricante maximiza la eficiencia mediante la
reutilización, y el cliente percibe un producto personalizado.

\subsection{Configuradores cotidianos: comida rápida
🥗🍔}\label{configuradores-cotidianos-comida-ruxe1pida}

El mismo principio aparece en contextos mucho más cercanos.

Ensaladas personalizadas: Un restaurante no ofrece cientos de ensaladas
distintas, sino un inventario común de ingredientes (base, proteínas,
toppings) que el cliente combina a su gusto.

Hamburguesas configurables: Cadenas como Five Guys ofrecen una base
estandarizada (pan y carne) y una lista de extras que permiten generar
múltiples variantes sin cambiar el proceso productivo.

En ambos casos:

el productor reutiliza los mismos componentes,

el cliente obtiene un producto ``a medida''.

\section*{Referencias}\label{referencias-2}

\markright{Referencias}

\phantomsection\label{refs--3}


\backmatter


\end{document}
